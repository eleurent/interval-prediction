\documentclass[slideopt,A4,showboxes,svgnames]{beamer}

%% list of packages here
\usepackage[absolute,showboxes,overlay]{textpos}
\usepackage{amssymb}
\usepackage{amsmath}
\usepackage{mathtools}

\usepackage{theme/beamerthemeinria}
%\usepackage{theme/beamerthemeinria2}
%\usepackage{theme/beamerthemeinria3}

\input{mathdef}
%\input{theme/style}

\title[Interval Prediction with Parametric Uncertainties]{Interval Prediction for \\ Continuous-Time Systems with \\ Parametric Uncertainties}
\subtitle{$^1$ Inria Valse, Lille, France\\
$^2$ CNAM, Paris, France}
\date[December, 2019]{December, 2019}
\author[Denis Efimov]{Edouard Leurent$^1$, \textbf{Denis Efimov}$^1$, \\Tarek Ra\"issi$^2$, Wilfrid Perruquetti$^1$}


\begin{document}

\begin{frame}
    \titlepage
\end{frame}

\frame{\tocpage}
 
\section{Problem statement}

\frame{\sectionpage}

\begin{frame}{The setting}
\begin{block}{Linear Parameter-Varying systems}
	\begin{equation*}
	\dot{x}(t)=A(\theta(t))x(t)+Bd(t)\label{eq:LPV_syst}
	\end{equation*}
	There are two sources of uncertainty:
	\begin{itemize}
		\item Parametric uncertainty $\theta(t)$
		\item External perturbations $d(t)$
	\end{itemize}
\end{block}

\centering
\includegraphics[width=0.45\textwidth]{img/interval-hull-0}
\end{frame}

\begin{frame}{The goal}
\begin{block}{Interval Prediction}
	Can we design an interval predictor $[\ux, \ox]$ that verifies:
	\begin{itemize}
		\item inclusion property: $\forall t, \ux \leq x(t) \leq \ox$;
		\item stable dynamics?
	\end{itemize}
	We want the predictor to be as tight as possible.
\end{block}

\centering
\includegraphics[width=0.45\textwidth]{img/interval-hull}
\end{frame}

\begin{frame}{Assumptions}
\begin{assumption}[Bounded trajectories]
	\begin{itemize}
		\item $\|x\|_{\infty} < \infty$
		\item $x(0)\in[\underline{x}_{0},\overline{x}_{0}]$ for some \alert{known} $\underline{x}_{0},\overline{x}_{0}\in\Real^{n}$
	\end{itemize}
\end{assumption}
\pause
\begin{assumption}[Bounded parameters]
	\begin{itemize}
		\item $\theta(t)\in\Theta$ for some \alert{known} $\Theta$
		\item The matrix function $A(\theta)$ is \alert{known}
	\end{itemize}
	\end{assumption}
\pause
\begin{assumption}[Bounded perturbations]
	\begin{itemize}
		\item $d(t)\in[\underline{d}(t),\overline{d}(t)]$ for some \alert{known} signals $\underline{d},\overline{d}\in\cL_{\infty}^{n}$
	\end{itemize}
\end{assumption}

\begin{flushright}
	How to proceed?
\end{flushright}
\end{frame}

\begin{frame}{A first idea}

Assume that $\underline{x}(t)\le x(t)\le\overline{x}(t)$, for some $t\geq0$. 
\pause
\begin{itemize}[<+->]
	\item[$\drsh$] To propagate the interval to $x(t+dt)$, we need to \\ \alert{bound $A(\theta(t))x(t)$}.
	\item[$\drsh$] Why not use \alert{interval arithmetics}?
\end{itemize}

\only<1-4>{\visible<4>{
\begin{lemma}[Image of an interval]
If $A$ a \alert{known} matrix, then
\begin{equation*}
A^{+}\underline{x}-A^{-}\overline{x}\le Ax\le A^{+}\overline{x}-A^{-}\underline{x}.\label{eq:Interval1}
\end{equation*}
where $A^+ = \max(A, 0)$ and $A^- = A-A^+$. 
\end{lemma}
}}
\only<5-6>{
\begin{lemma}[Product of intervals]
	If $A$ is \alert{unknown} but \alert{bounded} \textup{$\underline{A}\le A\le\overline{A}$},
	\begin{gather*}
	\underline{A}^{+}\underline{x}^{+}-\overline{A}^{+}\underline{x}^{-}-\underline{A}^{-}\overline{x}^{+}+\overline{A}^{-}\overline{x}^{-}\leq Ax\\
	\leq\overline{A}^{+}\overline{x}^{+}-\underline{A}^{+}\overline{x}^{-}-\overline{A}^{-}\underline{x}^{+}+\underline{A}^{-}\underline{x}^{-}. 
	\end{gather*}
\end{lemma}
}
\visible<6>{Since $A(\theta)$ and the set $\Theta$ are known, \\
	we can easily compute such bounds $\underline{A} \leq A(\theta(t))\leq \overline{A}$}
\end{frame}


\begin{frame}{A candidate predictor}
Following this result, define the predictor:
\begin{eqnarray}
\dot{\underline{x}}(t) & = & \underline{A}^{+}\underline{x}^{+}(t)-\overline{A}^{+}\underline{x}^{-}(t)-\underline{A}^{-}\overline{x}^{+}(t)\nonumber \\
&  & +\overline{A}^{-}\overline{x}^{-}(t)+B^{+}\underline{d}(t)-B^{-}\overline{d}(t),\label{eq:IP_direct}\\
\dot{\overline{x}}(t) & = & \overline{A}^{+}\overline{x}^{+}(t)-\underline{A}^{+}\overline{x}^{-}(t)-\overline{A}^{-}\underline{x}^{+}(t)\nonumber \\
&  & +\underline{A}^{-}\underline{x}^{-}(t)+B^{+}\overline{d}(t)-B^{-}\underline{d}(t),\nonumber \\
&  & \underline{x}(0)=\underline{x}_{0},\;\overline{x}(0)=\overline{x}_{0},\nonumber
\end{eqnarray}
\pause
\begin{proposition}[Inclusion property]
	\begin{itemize}
		\item[\checkmark] The predictor \eqref{eq:IP_direct} satisfies $\ux\leq x(t)\leq \ox(t)$
	\end{itemize} 
\end{proposition}
\pause
\begin{itemize}
	\item[?] But is it stable?
\end{itemize}
\end{frame}
\begin{frame}{Motivating example}
Consider the scalar system, for all $t\geq0$:
\[
	\dot{x}(t)=-\theta(t)x(t)+d(t), \text{ where} 
	\begin{cases}
	x(0)\in[\underline{x}_{0},\overline{x}_{0}]=[1.0, 1.1],\\
	\theta(t)\in\Theta=[\underline{\theta},\overline{\theta}]=[1,2],\\
	d(t)\in[\underline{d},\overline{d}]=[-0.1,0.1],
	\end{cases}
\]


\end{frame}

\section{Proposed predictor}
\frame{\sectionpage}


\begin{frame}{Our proposed predictor}
    \begin{alertblock}{}
 \begin{itemize}
    \item{Item 1 }
    \item {Item 2}
    \item {Item 3}
    \end{itemize}
  \end{alertblock}
      \begin{alertblock}{Avec titre}
 \begin{itemize}
    \item{Item 1 }
    \item {Item 2}
    \item {Item 3}
    \end{itemize}
  \end{alertblock}
\end{frame}

%Chapitre 3
\section{Application to autonomous driving}
 \frame{\sectionpage}

\begin{frame}{Deux autres exemples de blocs}
    \begin{exampleblock}{}
 \begin{itemize}
    \item{Item 1 }
    \item {Item 2}
    \item {Item 3}
    \end{itemize}
  \end{exampleblock}
      \begin{exampleblock}{Avec titre}
 \begin{itemize}
    \item{Item 1 }
    \item {Item 2}
    \item {Item 3}
    \end{itemize}
  \end{exampleblock}
\end{frame}
\begin{frame}{Titre}
Du \textcolor{gris_fonce_inria}{texte}  {en}  \textcolor{rouge_inria}{couleur} 
\end{frame}
\begin{frame}{Conclusion}
 
\end{frame}
\end{document}
