\documentclass[slideopt,A4,showboxes,svgnames]{beamer}

%% list of packages here
\usepackage[absolute,showboxes,overlay]{textpos}
\usepackage{amssymb}
\usepackage{pifont}
\usepackage{amsmath}
\usepackage{mathtools}
\usepackage{multicol}
\usepackage{blkarray}
\usepackage{booktabs}
\usepackage{multirow}
\usepackage[backend=biber, citestyle=authoryear, maxcitenames=2, maxbibnames=20]{biblatex}
\bibliography{../interval_lpv.bib}

\usepackage{theme/beamerthemeinria}
%\usepackage{theme/beamerthemeinria2}
%\usepackage{theme/beamerthemeinria3}

\input{mathdef}
\input{theme/style}

\title[Interval Prediction with Parametric Uncertainties]{Interval Prediction for \\ Continuous-Time Systems with \\ Parametric Uncertainties}
\date[December, 2019]{December, 2019}
\author[Edouard Leurent]{\textbf{Edouard Leurent}$^{1,2}$, {Denis Efimov}$^1$, \\Tarek Ra\"issi$^3$, Wilfrid Perruquetti$^4$}
\institute{$^1$ Inria, Lille, France\\
	$^2$ Renault Group, Guyancourt, France\\
	$^3$ CNAM, Paris, France\\
	$^4$ Centrale Lille, France}

\begin{document}

\begin{frame}
    \titlepage
\end{frame}

\frame{\tocpage}
 
\section{Problem statement}

\frame{\sectionpage}

\begin{frame}{Motivation}

We are interested in \alert{trajectory planning} for an autonomous vehicle.
 
 \begin{center}
 \href{https://github.com/eleurent/highway-env\#highway}{
 \includegraphics[width=0.7\textwidth]{img/highway-env}}    
 \end{center}

\begin{enumerate}
    \item We need to \alert{predict} the behaviours of other drivers
    \item These behaviours are {\red uncertain} and {\red non-linear}
\end{enumerate}
\vspace*{\baselineskip}
In order to efficiently capture model uncertainty, we consider the modelling framework of \alert{Linear Parameter-Varying} systems.

\end{frame}

\begin{frame}{The setting}
\begin{block}{Linear Parameter-Varying systems}
	\begin{equation*}
	\dot{x}(t)=A(\theta(t))x(t)+Bd(t)\label{eq:LPV_syst}
	\end{equation*}
	There are two sources of uncertainty:
	\begin{itemize}
		\item Parametric uncertainty $\theta(t)$
		\item External perturbations $d(t)$
	\end{itemize}
\end{block}

\centering
\includegraphics[width=0.45\textwidth]{img/interval-hull-0}
\end{frame}

\begin{frame}{The goal}
\begin{block}{Interval Prediction}
	Can we design an interval predictor $[\ux, \ox]$ that verifies:
	\vspace*{0.175cm}
	\begin{itemize}
		\item inclusion property: $\forall t, \ux \leq x(t) \leq \ox$;
		\item stable dynamics?
	\end{itemize}
	\vspace*{0.175cm}
	We want the predictor to be as tight as possible.

\end{block}

\centering
\includegraphics[width=0.45\textwidth]{img/interval-hull}
\end{frame}

\begin{frame}{Assumptions}
\begin{assumption}[Bounded trajectories]
	\begin{itemize}
		\item $\|x\|_{\infty} < \infty$
		\item $x(0)\in[\underline{x}_{0},\overline{x}_{0}]$ for some \alert{known} $\underline{x}_{0},\overline{x}_{0}\in\Real^{n}$
	\end{itemize}
\end{assumption}
\pause
\begin{assumption}[Bounded parameters]
	\begin{itemize}
		\item $\theta(t)\in\Theta$ for some \alert{known} $\Theta$
		\item The matrix function $A(\theta)$ is \alert{known}
	\end{itemize}
	\end{assumption}
\pause
\begin{assumption}[Bounded perturbations]
	\begin{itemize}
		\item $d(t)\in[\underline{d}(t),\overline{d}(t)]$ for some \alert{known} signals $\underline{d},\overline{d}\in\cL_{\infty}^{n}$
	\end{itemize}
\end{assumption}

\begin{flushright}
	How to proceed?
\end{flushright}
\end{frame}

\begin{frame}{A first idea}

Assume that $\underline{x}(t)\le x(t)\le\overline{x}(t)$, for some $t\geq0$. 
\pause
\begin{itemize}[<+->]
	\item[$\drsh$] To propagate the interval to $x(t+dt)$, we need to \\ \alert{bound $A(\theta(t))x(t)$}.
	\item[$\drsh$] Why not use \alert{interval arithmetics}?
\end{itemize}

\only<1-4>{\visible<4>{
\begin{lemma}[Image of an interval (\cite{EFRZS12})]
If $A$ a \alert{known} matrix, then
\begin{equation*}
A^{+}\underline{x}-A^{-}\overline{x}\le Ax\le A^{+}\overline{x}-A^{-}\underline{x}.\label{eq:Interval1}
\end{equation*}
where $A^+ = \max(A, 0)$ and $A^- = A-A^+$. 
\end{lemma}
}}
\only<5-6>{
\begin{lemma}[Product of intervals (\cite{EFRZS12})]
	If $A$ is \alert{unknown} but \alert{bounded} \textup{$\underline{A}\le A\le\overline{A}$},
	\begin{gather*}
	\underline{A}^{+}\underline{x}^{+}-\overline{A}^{+}\underline{x}^{-}-\underline{A}^{-}\overline{x}^{+}+\overline{A}^{-}\overline{x}^{-}\leq Ax\\
	\leq\overline{A}^{+}\overline{x}^{+}-\underline{A}^{+}\overline{x}^{-}-\overline{A}^{-}\underline{x}^{+}+\underline{A}^{-}\underline{x}^{-}. 
	\end{gather*}
\end{lemma}
}
\visible<6>{
	\begin{itemize}
		\item[\green \checkmark] Since $A(\theta)$ and the set $\Theta$ are known, \\
		we can easily compute such bounds $\underline{A} \leq A(\theta(t))\leq \overline{A}$
	\end{itemize}
}
\end{frame}


\begin{frame}{A candidate predictor}
Following this result, define the predictor:
\begin{eqnarray}
\dot{\underline{x}}(t) & = & \underline{A}^{+}\underline{x}^{+}(t)-\overline{A}^{+}\underline{x}^{-}(t)-\underline{A}^{-}\overline{x}^{+}(t)\nonumber \\
&  & +\overline{A}^{-}\overline{x}^{-}(t)+B^{+}\underline{d}(t)-B^{-}\overline{d}(t),\label{eq:IP_direct}\\
\dot{\overline{x}}(t) & = & \overline{A}^{+}\overline{x}^{+}(t)-\underline{A}^{+}\overline{x}^{-}(t)-\overline{A}^{-}\underline{x}^{+}(t)\nonumber \\
&  & +\underline{A}^{-}\underline{x}^{-}(t)+B^{+}\overline{d}(t)-B^{-}\underline{d}(t),\nonumber \\
&  & \underline{x}(0)=\underline{x}_{0},\;\overline{x}(0)=\overline{x}_{0},\nonumber
\end{eqnarray}
\pause
\begin{proposition}[Inclusion property]
	\begin{itemize}
		\item[\checkmark] The predictor \eqref{eq:IP_direct} satisfies $\ux\leq x(t)\leq \ox(t)$
	\end{itemize} 
\end{proposition}
\pause
\begin{itemize}
	\item[?] But is it stable?
\end{itemize}
\end{frame}
\begin{frame}{Motivating example}
Consider the scalar system, for all $t\geq0$:
\[
	\dot{x}(t)=-\theta(t)x(t)+d(t), \text{ where} 
	\begin{cases}
	x(0)\in[\underline{x}_{0},\overline{x}_{0}]=[1.0, 1.1],\\
	\theta(t)\in\Theta=[\underline{\theta},\overline{\theta}]=[1,2],\\
	d(t)\in[\underline{d},\overline{d}]=[-0.1,0.1],
	\end{cases}
\]
\begin{center}
\includegraphics<1>[trim={0 1.4cm 0 0.4cm}, clip, width=0.7\linewidth]{img/system}
\includegraphics<2>[trim={0 1.4cm 0 0.4cm}, clip, width=0.7\linewidth]{img/observer}
\end{center}

\begin{columns}
	\begin{column}{0.48\linewidth}
		\begin{itemize}
			\item[{\green \checkmark}] The system is always {\green stable}
		\end{itemize}
	\end{column}
	\begin{column}{0.52\linewidth}
		\begin{itemize}
			\item<2>[{\red \xmark}] The predictor \eqref{eq:IP_direct} is {\red unstable}
		\end{itemize}
	\end{column}
\end{columns}

\end{frame}

\section{Our proposed predictor}
\frame{\sectionpage}


\begin{frame}{Additional assumption}
\begin{assumption}[Polytopic Structure]
	\label{ass:a3} There exist $A_{0}$ \alert{Metzler} and $\Delta A_{0}, \cdots, \Delta A_{N}$ such that:
	\begin{gather*}
	A(\theta)=\underbrace{A_{0}}_{\substack{\text{Nominal}\\\text{dynamics}}} + \sum_{i=1}^{N}\lambda_{i}(\theta)\Delta A_{i},\quad \sum_{i=1}^{N}\underbrace{\lambda_{i}(\theta)}_{\geq 0}=1;\quad \forall\theta\in\Theta
	\end{gather*}
\end{assumption}
\centering
\includegraphics[width=0.42\linewidth]{img/polytope}
\end{frame}

\begin{frame}{Our proposed predictor}
Denote
\[
\Delta A_{+}=\sum_{i=1}^{N}\Delta A_{i}^{+},\;\Delta A_{-}=\sum_{i=1}^{N}\Delta A_{i}^{-},
\]
We define the predictor
\begin{eqnarray}
\alert{\dot{\underline{x}}(t)} & = & A_{0}\alert{\underline{x}(t)}-\Delta A_{+}\underline{x}^{-}(t)-\Delta A_{-}\overline{x}^{+}(t)\nonumber \\
&  & +B^{+}\underline{d}(t)-B^{-}\overline{d}(t),\nonumber\\
\alert{\dot{\overline{x}}(t)} & = & A_{0}\alert{\overline{x}(t)}+\Delta A_{+}\overline{x}^{+}(t)+\Delta A_{-}\underline{x}^{-}(t)\label{eq:IP_main} \\
&  & +B^{+}\overline{d}(t)-B^{-}\underline{d}(t),\nonumber \\
&  & \underline{x}(0)=\underline{x}_{0},\;\overline{x}(0)=\overline{x}_{0}\nonumber 
\end{eqnarray}


\begin{theorem}[Inclusion property]
	\label{thm:main}
	The predictor \eqref{eq:IP_main} ensures $\ux\leq x(t)\leq \ox$. 
\end{theorem}

\end{frame}

\begin{frame}{Stability}

\begin{theorem}[Stability]
	If there exist diagonal matrices $P$, $Q$, $Q_{+}$, $Q_{-}$, $Z_{+}$, $Z_{-}$, $\Psi_{+}$, $\Psi_{-}$, $\Psi$, $\Gamma\in\Real^{2n\times2n}$ such that the following LMIs are satisfied:
	\begin{gather*}
	P+\min\{Z_{+},Z_{-}\}>0,\;\hleb{\Upsilon\preceq0},\;\Gamma>0,\\
	Q+\min\{Q_{+},Q_{-}\}+2\min\{\Psi_{+},\Psi_{-}\}>0,
	\end{gather*}
	where \hlb{$\Upsilon = \Upsilon(A_0, \Delta A_-, \Delta A_+, \Psi_-, \Psi_+, \Psi)$}, 
	
	then the predictor \eqref{eq:IP_main} is input-to-state stable with respect to the inputs $\underline{d}$, $\overline{d}$.
\end{theorem}

\end{frame}

% \begin{frame}
% where{\footnotesize{}
% 	\begin{gather*}
% 	\Upsilon=\left[\begin{array}{cccc}
% 	\Upsilon_{11} & \Upsilon_{12} & \Upsilon_{13} & P\\
% 	\Upsilon_{12}^{\top} & \Upsilon_{22} & \Upsilon_{23} & Z_{+}\\
% 	\Upsilon_{13}^{\top} & \Upsilon_{23}^{\top} & \Upsilon_{33} & Z_{-}\\
% 	P & Z_{+} & Z_{-} & -\Gamma
% 	\end{array}\right],\\
% 	\Upsilon_{11}=\mathcal{A}^{\top}P+P\mathcal{A}+Q,\;\Upsilon_{12}=\mathcal{A}^{\top}Z_{+}+PR_{+}+\Psi_{+},\\
% 	\Upsilon_{13}=\mathcal{A}^{\top}Z_{-}+PR_{-}+\Psi_{-},\;\Upsilon_{22}=Z_{+}R_{+}+R_{+}^{\top}Z_{+}+Q_{+},\\
% 	\Upsilon_{23}=Z_{+}R_{-}+R_{+}^{\top}Z_{-}+\Psi,\;\Upsilon_{33}=Z_{-}R_{-}+R_{-}^{\top}Z_{-}+Q_{-},\\
% 	\mathcal{A}=\left[\begin{array}{cc}
% 	A_{0} & 0\\
% 	0 & A_{0}
% 	\end{array}\right],\;R_{+}=\left[\begin{array}{cc}
% 	0 & -\Delta A_{-}\\
% 	0 & \Delta A_{+}
% 	\end{array}\right],\;R_{-}=\left[\begin{array}{cc}
% 	\Delta A_{+} & 0\\
% 	-\Delta A_{-} & 0
% 	\end{array}\right],
% 	\end{gather*}
% }
% \end{frame}

\begin{frame}{Sketch of proof}
\begin{enumerate}
	\item Define the extended state vector as $X=[\underline{x}^{\top}\;\;\overline{x}^{\top}]^{\top}$
	\item It follows the dynamics $$\dot{X}(t)=\mathcal{A}X(t)+R_{+}X^{+}(t)-R_{-}X^{-}(t)+\delta(t)$$
	$$\mathcal{A}=\left[\begin{array}{cc}
    A_{0} & 0\\
    0 & A_{0}
    \end{array}\right]
    	\;R_{+}=\left[\begin{array}{cc}
    0 & -\Delta A_{-}\\
    0 & \Delta A_{+}
    \end{array}\right],\;R_{-}=\left[\begin{array}{cc}
    \Delta A_{+} & 0\\
    -\Delta A_{-} & 0
    \end{array}\right]$$
	\item Consider a candidate Lyapunov function:
	\begin{gather*}
	V(X)=X^{\top}PX+X{}^{\top}Z_{+}X^{+}-X^{\top}Z_{-}X^{-}
	\end{gather*}
	\item $V(X)$ is positive definite provided that
	$
	P+\min\{Z_{+},Z_{-}\}>0,
	$
	\item Check on which condition we have $\dot{V}(X) \leq 0$
\end{enumerate}
\end{frame}

\begin{frame}{Back to our motivating example}
Recall the scalar system:
\[
\dot{x}(t)=-\theta(t)x(t)+d(t), \text{ where} 
\begin{cases}
x(0)\in[\underline{x}_{0},\overline{x}_{0}]=[1.0, 1.1],\\
\theta(t)\in\Theta=[\underline{\theta},\overline{\theta}]=[1,2],\\
d(t)\in[\underline{d},\overline{d}]=[-0.1,0.1],
\end{cases}
\]
\begin{center}
	\includegraphics<1>[trim={0 1.4cm 0 0.4cm}, clip, width=0.7\linewidth]{../img/predictor}
\end{center}

\begin{columns}
	\begin{column}{0.48\linewidth}
		\begin{itemize}
			\item[{\green \checkmark}] The system is always {\green stable}
		\end{itemize}
	\end{column}
	\begin{column}{0.52\linewidth}
		\begin{itemize}
			\item[{\green \checkmark}] The predictor \eqref{eq:IP_main} is {\green stable}
		\end{itemize}
	\end{column}
\end{columns}

\end{frame}


\section{Application to autonomous driving}

 \frame{\sectionpage}
 
 \begin{frame}{A multi-agent system}
 
 \begin{center}
 \href{https://github.com/eleurent/highway-env\#highway}{
 \includegraphics[width=0.7\textwidth]{img/highway-env}}    
 \end{center}
 
 $$\dot{z}_i=f_i(Z,\theta_i),\;i=\overline{1,N},$$ where
 
 \begin{itemize}
 	\item $z_i = [x_i, y_i, v_i, \psi_i]^\top\in\Real^4$ is the state of an agent
 	\item $\theta_i\in\Real^5$ is the corresponding unknown behavioural parameters
 	\item $Z = [z_1,\dots,z_N]^\top\in\Real^{4N}$  is the joint state of the traffic
 	\item $\theta=[\theta_1,\dots,\theta_N]^\top \in \Pi\subset\Real^{5N}$	
 \end{itemize}
\end{frame}

\begin{frame}{Kinematics}
\begin{center}
\begin{tabular}{ccl}
    \toprule
    \multirow{3}{*}{States} & $(x_i, y_i)$ & position \\
    & $v_i$ & longitudinal velocity \\
    & $\psi_i$ & yaw angle \\
    \midrule
    \multirow{2}{*}{Controls} & $a_i$ & longitudinal acceleration \\
    & $\beta_i$ & slip angle at the center of gravity \\
    \bottomrule
\end{tabular}
\end{center}
Each vehicle follows the Kinematic Bicycle Model:
\begin{align}
	\dot{x}_i &= v_i\cos(\psi_i), \nonumber\\
	\dot{y}_i &= v_i\sin(\psi_i), \nonumber\\
	\dot{v}_i &= a_i, \nonumber\\
	\dot{\psi}_i &= \frac{v_i}{l}tan(\beta_i), \nonumber
\end{align}
\end{frame}

\begin{frame}{Longitudinal control}
A linear controller using three features inspired from the intelligent driver model (IDM) [\cite{Treiber2000}].

	\begin{equation*}
	a_i = \begin{bmatrix}
	\theta_{i,1} & \theta_{i,2} & \theta_{i,3}
	\end{bmatrix} \begin{bmatrix}
	v_0 - v_i \\
	-(v_{f_i}-v_i)^- \\
	-(x_{f_i} - x_i - (d_0 + v_iT))^- \\
	\end{bmatrix},
	\label{eq:theta_a}
	\end{equation*}
	%where $v_0, d_0$ and $T$ respectively denote the speed limit, jam distance and time gap given by traffic rules, and $f_i$ is the index of the leading vehicle of vehicle $i$.
	where
	\begin{center}
	\begin{tabular}{cl}
    \toprule
    $v_0$ & speed limit \\
    $d_0$ & jam distance \\
    $T$ & time gap \\
    $f_i$ & index of vehicle $i$'s front vehicle\\
    \bottomrule
    \end{tabular}    
	\end{center}
	
\end{frame}

\begin{frame}{Lateral Control}
A cascade controller of lateral position $y_i$ and heading $\psi_i$:
\begin{align}
\label{eq:heading-command}
\dot{\psi}_i &= \theta_{i,5}\left(\psi_{L_i}+\sin^{-1}\left(\frac{\tilde{v}_{i,y}}{v_i}\right)-\psi_i\right),\\
\tilde{v}_{i,y} &= \theta_{i,4} (y_{L_i}-y_i). \nonumber
\end{align}
We assume that the drivers choose their steering command $\beta_i$ such that \eqref{eq:heading-command} is always achieved: $\beta_i = \tan^{-1}(\frac{l}{v_i}\dot{\psi}_i)$.
\end{frame}

\begin{frame}{LPV Formulation}

We linearize trigonometric operators around $y_i=y_{L_i}$ and $\psi_i=\psi_{L_i}$.

This yields the following longitudinal dynamics:
\begin{align*}
\dot{x}_i &= v_i,\\
\dot v_i &= \theta_{i,1} (v_0 - v_i) + \theta_{i,2} (v_{f_i} - v_i) + \theta_{i,3}(x_{f_i} - x_i - d_0 - v_i T),
\end{align*}
where $\theta_{i,2}$ and $\theta_{i,3}$ are set to $0$ whenever the corresponding features are not active.
\end{frame}

\begin{frame}{LPV Formulation}
$$\dot{Z} = A(\theta)(Z-Z_c) + d.$$ For example, in the case of two vehicles only:
\begin{equation*}
Z = \begin{bmatrix}
x_i \\
x_{f_i} \\
v_i \\
v_{f_i} \\
\end{bmatrix}
,\quad
Z_c = \begin{bmatrix}
-d_0-v_0 T \\
0 \\
v_0\\
v_0 \\
\end{bmatrix}
,\quad
d = \begin{bmatrix}
v_0 \\
v_0 \\
0\\
0\\
\end{bmatrix}
\end{equation*}

\begin{equation*}
A(\theta)
=
\begin{blockarray}{ccccc}
& i & f_i & i & f_i \\
\begin{block}{c[cccc]}
i & 0 & 0 & 1 & 0 \\
f_i & 0 & 0 & 0 & 1 \\
i & -\theta_{i,3} & \theta_{i,3} & -\theta_{i,1}-\theta_{i,2}-\theta_{i,3} & \theta_{i,2} \\
f_i & 0 & 0 & 0 & -\theta_{f_i,1} \\
\end{block}
\end{blockarray}
\end{equation*}
\end{frame}

\begin{frame}{LPV Formulation}
The lateral dynamics are in a similar form:
\begin{equation*}
\begin{bmatrix}
\dot{y}_i \\
\dot{\psi}_i \\
\end{bmatrix}
=
\begin{bmatrix}
0 & {\red v_i} \\
-\frac{\theta_{i,4} \theta_{i,5}}{\red v_i} & -\theta_{i,5}
\end{bmatrix}
\begin{bmatrix}
y_i - y_{L_i} \\
\psi_i - \psi_{L_i}
\end{bmatrix}
+
\begin{bmatrix}
v_i\psi_{L_i} \\
0
\end{bmatrix}
\end{equation*}
Here, the dependency in {\red $v_i$} is seen as an uncertain parametric dependency, \emph{i.e.} \alert{$\theta_{i,6}=v_i$}, with constant bounds assumed for $v_i$ using an overset of the longitudinal interval predictor.
\end{frame}

\begin{frame}{Results}
\centering
\vspace*{\baselineskip}
\href{https://eleurent.github.io/interval-prediction/\#direct-interval-predictor}{
\includegraphics[width=0.75\textwidth]{../img/driving_observer.png}}

The naive predictor \eqref{eq:IP_direct} quickly diverges

\href{https://eleurent.github.io/interval-prediction/\#novel-interval-predictor}{
\includegraphics[width=0.75\textwidth]{../img/driving_predictor.png}}

The proposed predictor \eqref{eq:IP_main} remains stable
\end{frame}

\begin{frame}{Results}
\centering
\vspace*{\baselineskip}
\href{https://eleurent.github.io/interval-prediction/\#novel-interval-predictor}{
\includegraphics[width=0.75\textwidth]{../img/lane_change_predictor.png}}

Prediction during a lane change maneuver

\href{https://eleurent.github.io/interval-prediction/\#right-hand-traffic}{
\includegraphics[width=0.75\textwidth]{../img/overtake.png}}

Prediction with uncertainty in the followed lane $L_i$
\end{frame}

\begin{frame}{Conclusion}
    
    \begin{block}{Problem formulation}
    \begin{itemize}
        \item Prediction of an {\red uncertain non-linear} system
        \item[\red $\drsh$] Within the \alert{LPV} framework
        \item[\red $\drsh$] Design of an \alert{interval} predictor $[\underline{x}(t), \overline{x}(t)]$?
    \end{itemize}
    \end{block}
    
    \begin{block}{Proposed solution}
    \begin{itemize}
        \item Direct prediction with interval arithmetics is \alert{valid} but {\red unstable}
        \item[\red $\drsh$] Assume \alert{polytopic uncertainty} structure around a nominal $A_0$
        \item[\red $\drsh$] \alert{Ensure stability} using a Luyapunov function in an LMI form
    \end{itemize}
    \end{block}
    
    \begin{block}{Application}
    \begin{itemize}
        \item Joint prediction of coupled traffic dynamics
        \item Can be used as a building block for robust planning
    \end{itemize}
    \end{block}
\end{frame}
\end{document}
